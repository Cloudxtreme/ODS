
%% bare_conf.tex
%% V1.3
%% 2007/01/11
%% by Michael Shell
%% See:
%% http://www.michaelshell.org/
%% for current contact information.
%%
%% This is a skeleton file demonstrating the use of IEEEtran.cls
%% (requires IEEEtran.cls version 1.7 or later) with an IEEE conference paper.
%%
%% Support sites:
%% http://www.michaelshell.org/tex/ieeetran/
%% http://www.ctan.org/tex-archive/macros/latex/contrib/IEEEtran/
%% and
%% http://www.ieee.org/

%%*************************************************************************
%% Legal Notice:
%% This code is offered as-is without any warranty either expressed or
%% implied; without even the implied warranty of MERCHANTABILITY or
%% FITNESS FOR A PARTICULAR PURPOSE! 
%% User assumes all risk.
%% In no event shall IEEE or any contributor to this code be liable for
%% any damages or losses, including, but not limited to, incidental,
%% consequential, or any other damages, resulting from the use or misuse
%% of any information contained here.
%%
%% All comments are the opinions of their respective authors and are not
%% necessarily endorsed by the IEEE.
%%
%% This work is distributed under the LaTeX Project Public License (LPPL)
%% ( http://www.latex-project.org/ ) version 1.3, and may be freely used,
%% distributed and modified. A copy of the LPPL, version 1.3, is included
%% in the base LaTeX documentation of all distributions of LaTeX released
%% 2003/12/01 or later.
%% Retain all contribution notices and credits.
%% ** Modified files should be clearly indicated as such, including  **
%% ** renaming them and changing author support contact information. **
%%
%% File list of work: IEEEtran.cls, IEEEtran_HOWTO.pdf, bare_adv.tex,
%%                    bare_conf.tex, bare_jrnl.tex, bare_jrnl_compsoc.tex
%%*************************************************************************

% *** Authors should verify (and, if needed, correct) their LaTeX system  ***
% *** with the testflow diagnostic prior to trusting their LaTeX platform ***
% *** with production work. IEEE's font choices can trigger bugs that do  ***
% *** not appear when using other class files.                            ***
% The testflow support page is at:
% http://www.michaelshell.org/tex/testflow/



% Note that the a4paper option is mainly intended so that authors in
% countries using A4 can easily print to A4 and see how their papers will
% look in print - the typesetting of the document will not typically be
% affected with changes in paper size (but the bottom and side margins will).
% Use the testflow package mentioned above to verify correct handling of
% both paper sizes by the user's LaTeX system.
%
% Also note that the "draftcls" or "draftclsnofoot", not "draft", option
% should be used if it is desired that the figures are to be displayed in
% draft mode.
%
\documentclass[conference]{IEEEtran}
% Add the compsoc option for Computer Society conferences.
%
% If IEEEtran.cls has not been installed into the LaTeX system files,
% manually specify the path to it like:
% \documentclass[conference]{../sty/IEEEtran}





% Some very useful LaTeX packages include:
% (uncomment the ones you want to load)


% *** MISC UTILITY PACKAGES ***
%
%\usepackage{ifpdf}
% Heiko Oberdiek's ifpdf.sty is very useful if you need conditional
% compilation based on whether the output is pdf or dvi.
% usage:
% \ifpdf
%   % pdf code
% \else
%   % dvi code
% \fi
% The latest version of ifpdf.sty can be obtained from:
% http://www.ctan.org/tex-archive/macros/latex/contrib/oberdiek/
% Also, note that IEEEtran.cls V1.7 and later provides a builtin
% \ifCLASSINFOpdf conditional that works the same way.
% When switching from latex to pdflatex and vice-versa, the compiler may
% have to be run twice to clear warning/error messages.






% *** CITATION PACKAGES ***
%
%\usepackage{cite}
% cite.sty was written by Donald Arseneau
% V1.6 and later of IEEEtran pre-defines the format of the cite.sty package
% \cite{} output to follow that of IEEE. Loading the cite package will
% result in citation numbers being automatically sorted and properly
% "compressed/ranged". e.g., [1], [9], [2], [7], [5], [6] without using
% cite.sty will become [1], [2], [5]--[7], [9] using cite.sty. cite.sty's
% \cite will automatically add leading space, if needed. Use cite.sty's
% noadjust option (cite.sty V3.8 and later) if you want to turn this off.
% cite.sty is already installed on most LaTeX systems. Be sure and use
% version 4.0 (2003-05-27) and later if using hyperref.sty. cite.sty does
% not currently provide for hyperlinked citations.
% The latest version can be obtained at:
% http://www.ctan.org/tex-archive/macros/latex/contrib/cite/
% The documentation is contained in the cite.sty file itself.






% *** GRAPHICS RELATED PACKAGES ***
%
\ifCLASSINFOpdf
   \usepackage{graphicx}
  % declare the path(s) where your graphic files are
  % \graphicspath{{C:\Users\Administrator\Pictures}}
  % and their extensions so you won't have to specify these with
  % every instance of \includegraphics
   \DeclareGraphicsExtensions{.pdf,.jpeg,.png}
\else
  % or other class option (dvipsone, dvipdf, if not using dvips). graphicx
  % will default to the driver specified in the system graphics.cfg if no
  % driver is specified.
  % \usepackage[dvips]{graphicx}
  % declare the path(s) where your graphic files are
  % \graphicspath{{../eps/}}
  % and their extensions so you won't have to specify these with
  % every instance of \includegraphics
  % \DeclareGraphicsExtensions{.eps}
\fi
% graphicx was written by David Carlisle and Sebastian Rahtz. It is
% required if you want graphics, photos, etc. graphicx.sty is already
% installed on most LaTeX systems. The latest version and documentation can
% be obtained at: 
% http://www.ctan.org/tex-archive/macros/latex/required/graphics/
% Another good source of documentation is "Using Imported Graphics in
% LaTeX2e" by Keith Reckdahl which can be found as epslatex.ps or
% epslatex.pdf at: http://www.ctan.org/tex-archive/info/
%
% latex, and pdflatex in dvi mode, support graphics in encapsulated
% postscript (.eps) format. pdflatex in pdf mode supports graphics
% in .pdf, .jpeg, .png and .mps (metapost) formats. Users should ensure
% that all non-photo figures use a vector format (.eps, .pdf, .mps) and
% not a bitmapped formats (.jpeg, .png). IEEE frowns on bitmapped formats
% which can result in "jaggedy"/blurry rendering of lines and letters as
% well as large increases in file sizes.
%
% You can find documentation about the pdfTeX application at:
% http://www.tug.org/applications/pdftex





% *** MATH PACKAGES ***
%
%\usepackage[cmex10]{amsmath}
% A popular package from the American Mathematical Society that provides
% many useful and powerful commands for dealing with mathematics. If using
% it, be sure to load this package with the cmex10 option to ensure that
% only type 1 fonts will utilized at all point sizes. Without this option,
% it is possible that some math symbols, particularly those within
% footnotes, will be rendered in bitmap form which will result in a
% document that can not be IEEE Xplore compliant!
%
% Also, note that the amsmath package sets \interdisplaylinepenalty to 10000
% thus preventing page breaks from occurring within multiline equations. Use:
%\interdisplaylinepenalty=2500
% after loading amsmath to restore such page breaks as IEEEtran.cls normally
% does. amsmath.sty is already installed on most LaTeX systems. The latest
% version and documentation can be obtained at:
% http://www.ctan.org/tex-archive/macros/latex/required/amslatex/math/





% *** SPECIALIZED LIST PACKAGES ***
%
\usepackage{algorithmic}
\usepackage{algorithm}% http://ctan.org/pkg/algorithms
%\usepackage{algpseudocode}% http://ctan.org/pkg/algorithmicx
% algorithmic.sty was written by Peter Williams and Rogerio Brito.
% This package provides an algorithmic environment fo describing algorithms.
% You can use the algorithmic environment in-text or within a figure
% environment to provide for a floating algorithm. Do NOT use the algorithm
% floating environment provided by algorithm.sty (by the same authors) or
% algorithm2e.sty (by Christophe Fiorio) as IEEE does not use dedicated
% algorithm float types and packages that provide these will not provide
% correct IEEE style captions. The latest version and documentation of
% algorithmic.sty can be obtained at:
% http://www.ctan.org/tex-archive/macros/latex/contrib/algorithms/
% There is also a support site at:
% http://algorithms.berlios.de/index.html
% Also of interest may be the (relatively newer and more customizable)
% algorithmicx.sty package by Szasz Janos:
% http://www.ctan.org/tex-archive/macros/latex/contrib/algorithmicx/




% *** ALIGNMENT PACKAGES ***
%
%\usepackage{array}
% Frank Mittelbach's and David Carlisle's array.sty patches and improves
% the standard LaTeX2e array and tabular environments to provide better
% appearance and additional user controls. As the default LaTeX2e table
% generation code is lacking to the point of almost being broken with
% respect to the quality of the end results, all users are strongly
% advised to use an enhanced (at the very least that provided by array.sty)
% set of table tools. array.sty is already installed on most systems. The
% latest version and documentation can be obtained at:
% http://www.ctan.org/tex-archive/macros/latex/required/tools/


%\usepackage{mdwmath}
%\usepackage{mdwtab}
% Also highly recommended is Mark Wooding's extremely powerful MDW tools,
% especially mdwmath.sty and mdwtab.sty which are used to format equations
% and tables, respectively. The MDWtools set is already installed on most
% LaTeX systems. The lastest version and documentation is available at:
% http://www.ctan.org/tex-archive/macros/latex/contrib/mdwtools/


% IEEEtran contains the IEEEeqnarray family of commands that can be used to
% generate multiline equations as well as matrices, tables, etc., of high
% quality.


%\usepackage{eqparbox}
% Also of notable interest is Scott Pakin's eqparbox package for creating
% (automatically sized) equal width boxes - aka "natural width parboxes".
% Available at:
% http://www.ctan.org/tex-archive/macros/latex/contrib/eqparbox/





% *** SUBFIGURE PACKAGES ***
%\usepackage[tight,footnotesize]{subfigure}
% subfigure.sty was written by Steven Douglas Cochran. This package makes it
% easy to put subfigures in your figures. e.g., "Figure 1a and 1b". For IEEE
% work, it is a good idea to load it with the tight package option to reduce
% the amount of white space around the subfigures. subfigure.sty is already
% installed on most LaTeX systems. The latest version and documentation can
% be obtained at:
% http://www.ctan.org/tex-archive/obsolete/macros/latex/contrib/subfigure/
% subfigure.sty has been superceeded by subfig.sty.



%\usepackage[caption=false]{caption}
%\usepackage[font=footnotesize]{subfig}
% subfig.sty, also written by Steven Douglas Cochran, is the modern
% replacement for subfigure.sty. However, subfig.sty requires and
% automatically loads Axel Sommerfeldt's caption.sty which will override
% IEEEtran.cls handling of captions and this will result in nonIEEE style
% figure/table captions. To prevent this problem, be sure and preload
% caption.sty with its "caption=false" package option. This is will preserve
% IEEEtran.cls handing of captions. Version 1.3 (2005/06/28) and later 
% (recommended due to many improvements over 1.2) of subfig.sty supports
% the caption=false option directly:
%\usepackage[caption=false,font=footnotesize]{subfig}
%
% The latest version and documentation can be obtained at:
% http://www.ctan.org/tex-archive/macros/latex/contrib/subfig/
% The latest version and documentation of caption.sty can be obtained at:
% http://www.ctan.org/tex-archive/macros/latex/contrib/caption/




% *** FLOAT PACKAGES ***
%
%\usepackage{fixltx2e}
% fixltx2e, the successor to the earlier fix2col.sty, was written by
% Frank Mittelbach and David Carlisle. This package corrects a few problems
% in the LaTeX2e kernel, the most notable of which is that in current
% LaTeX2e releases, the ordering of single and double column floats is not
% guaranteed to be preserved. Thus, an unpatched LaTeX2e can allow a
% single column figure to be placed prior to an earlier double column
% figure. The latest version and documentation can be found at:
% http://www.ctan.org/tex-archive/macros/latex/base/



%\usepackage{stfloats}
% stfloats.sty was written by Sigitas Tolusis. This package gives LaTeX2e
% the ability to do double column floats at the bottom of the page as well
% as the top. (e.g., "\begin{figure*}[!b]" is not normally possible in
% LaTeX2e). It also provides a command:
%\fnbelowfloat
% to enable the placement of footnotes below bottom floats (the standard
% LaTeX2e kernel puts them above bottom floats). This is an invasive package
% which rewrites many portions of the LaTeX2e float routines. It may not work
% with other packages that modify the LaTeX2e float routines. The latest
% version and documentation can be obtained at:
% http://www.ctan.org/tex-archive/macros/latex/contrib/sttools/
% Documentation is contained in the stfloats.sty comments as well as in the
% presfull.pdf file. Do not use the stfloats baselinefloat ability as IEEE
% does not allow \baselineskip to stretch. Authors submitting work to the
% IEEE should note that IEEE rarely uses double column equations and
% that authors should try to avoid such use. Do not be tempted to use the
% cuted.sty or midfloat.sty packages (also by Sigitas Tolusis) as IEEE does
% not format its papers in such ways.





% *** PDF, URL AND HYPERLINK PACKAGES ***
%
%\usepackage{url}
% url.sty was written by Donald Arseneau. It provides better support for
% handling and breaking URLs. url.sty is already installed on most LaTeX
% systems. The latest version can be obtained at:
% http://www.ctan.org/tex-archive/macros/latex/contrib/misc/
% Read the url.sty source comments for usage information. Basically,
% \url{my_url_here}.





% *** Do not adjust lengths that control margins, column widths, etc. ***
% *** Do not use packages that alter fonts (such as pslatex).         ***
% There should be no need to do such things with IEEEtran.cls V1.6 and later.
% (Unless specifically asked to do so by the journal or conference you plan
% to submit to, of course. )


% correct bad hyphenation here
\hyphenation{op-tical net-works semi-conduc-tor}


\begin{document}
%
% paper title
% can use linebreaks \\ within to get better formatting as desired
\title{Layer based routing to obtain quality optimization of video streaming in the Future Internet}


% author names and affiliations
% use a multiple column layout for up to three different
% affiliations
\author{\IEEEauthorblockN{Sander Vrijders, Piet Smet, Youri Flement, Bruno Chevalier}
\IEEEauthorblockA{Ghent University - iMinds, Department of Information Technology,\\ 
Gaston Crommenlaan 8 bus 201, 9050 Gent, Belgium}
}

% conference papers do not typically use \thanks and this command
% is locked out in conference mode. If really needed, such as for
% the acknowledgment of grants, issue a \IEEEoverridecommandlockouts
% after \documentclass

% for over three affiliations, or if they all won't fit within the width
% of the page, use this alternative format:
% 
%\author{\IEEEauthorblockN{Michael Shell\IEEEauthorrefmark{1},
%Homer Simpson\IEEEauthorrefmark{2},
%James Kirk\IEEEauthorrefmark{3}, 
%Montgomery Scott\IEEEauthorrefmark{3} and
%Eldon Tyrell\IEEEauthorrefmark{4}}
%\IEEEauthorblockA{\IEEEauthorrefmark{1}School of Electrical and Computer Engineering\\
%Georgia Institute of Technology,
%Atlanta, Georgia 30332--0250\\ Email: see http://www.michaelshell.org/contact.html}
%\IEEEauthorblockA{\IEEEauthorrefmark{2}Twentieth Century Fox, Springfield, USA\\
%Email: homer@thesimpsons.com}
%\IEEEauthorblockA{\IEEEauthorrefmark{3}Starfleet Academy, San Francisco, California 96678-2391\\
%Telephone: (800) 555--1212, Fax: (888) 555--1212}
%\IEEEauthorblockA{\IEEEauthorrefmark{4}Tyrell Inc., 123 Replicant Street, Los Angeles, California 90210--4321}}




% use for special paper notices
%\IEEEspecialpapernotice{(Invited Paper)}




% make the title area
\maketitle


\begin{abstract}
%\boldmath
In TCP/IP networks, there is no inherent support for Quality of Service.
Routing is done on the minimum hop count from sender to receiver, and all traffic is best-effort traffic. 
This is unsatisfactory for applications such as video streaming. 
This paper presents an OpenFlow controller application which provides 
Quality of Service (\textit{QoS}) support for video streaming using Scalable Video Coding (\textit{SVC}) in an OpenFlow network. 
First we route the base layer of SVC encoded videos as a 
high priority QoS flow while the enhancement layers are routed at a lower QoS or as a best-effort flow. 
Secondly, we provide failure recovery so the OpenFlow network can converge back to an optimal 
state when links fail. We show that using different costs to calculate paths for different flows 
can improve the overall throughput of video streams when network congestion and delay occur.
\end{abstract}
% IEEEtran.cls defaults to using nonbold math in the Abstract.
% This preserves the distinction between vectors and scalars. However,
% if the conference you are submitting to favors bold math in the abstract,
% then you can use LaTeX's standard command \boldmath at the very start
% of the abstract to achieve this. Many IEEE journals/conferences frown on
% math in the abstract anyway.

% no keywords




% For peer review papers, you can put extra information on the cover
% page as needed:
% \ifCLASSOPTIONpeerreview
% \begin{center} \bfseries EDICS Category: 3-BBND \end{center}
% \fi
%
% For peerreview papers, this IEEEtran command inserts a page break and
% creates the second title. It will be ignored for other modes.
\IEEEpeerreviewmaketitle



\section{Introduction}
In a TCP/IP-based network, each router or switch contains a flow table containing the 
next hop for each destination, based on the shortest path calculation. 
Media applications, such as video streaming, 
require minimal packet loss and minimal network delay. 
When these video flows are sent over the same path as other best effort traffic they 
are subject to network congestion and delay which renders an unsatisfactory outcome. 
Several efforts have been made to give precedence to certain types of traffic, defined as Quality of Service (QoS).
There are solutions available for QoS support in a TCP/IP-based network, 
such as DiffServ \cite{diffserv} and IntServ \cite{intserv}, but these 
solutions only work well in managed networks.
Tunneling with MPLS offers a partial solution, but it lacks real-time reconfigurability and adaptivity. \cite{mpls}

Scalable Video Coding (SVC) \cite{schwarz2007overview} encodes the stream in a base layer along with several enhancement layers. 
In order to obtain qualitative video streams, it is important that the base layer has no 
packet loss and minimal delay, which is hard to accomplish without support for QoS. 
To solve this issue we need to differentiate between the different flows and route these flows 
differently than normal best effort traffic.

We use OpenFlow \cite{mckeown2008openflow} in order to provide QoS support. 
OpenFlow is a protocol for switches, allowing to control the packet forwarding process.
This way, networks are not only configurable, but also programmable.
Each OpenFlow Switch has a flow table, which is used for packet lookup and 
forwarding, and a secure channel to an external controller (see Figure \ref{fig:openflow1}). 
The flow table consists of flow entries.
Each flow table entry matches a traffic pattern, by matching the header fields.
The header fields that can be matched in OpenFlow 1.0 can be seen in Table \ref{tab:matchedflows}.
The controller makes the network programmable by managing the switch over 
the secure channel using the OpenFlow protocol.
The controller installs new entries in the flow table to manage the network traffic.
Entries have a hard and a soft timeout. A soft timeout means the entry gets removed if it has not been used for a certain time.
A hard timeout means the flow will be removed after the timeout, even if it is still in use.
There are many implementations of controllers available for OpenFlow switches.
The implementation of the controller we will use is Floodlight \cite{floodlight}, because it is open source and widely used.

\begin{table}[htb]
\centering
\begin{tabular}{| c | *{11}{c|}}
\hline 
In Port & Ethernet Source Address & Ethernet Destination Address \\ \hline
Ethernet Type & VLAN ID & VLAN Priority \\ \hline
IP Source & IP Destination & IP Protocol \\ \hline
IP ToS Bits & TCP/UDP Source Port & TCP/UDP Destination Port\\ \hline 
\end{tabular}
\caption{Fields that can be matched in OpenFlow 1.0}
\label{tab:matchedflows}
\end{table}

\begin{figure}[htb]
\centering
\includegraphics[scale=0.6]{fig/openflow1}
\caption{How OpenFlow works}
\label{fig:openflow1}
\end{figure}

The rest of this paper is organized as follows.
In Section II of this paper we compare related work to our own. 
In section III we take a closer look at the design considerations for 
developing an OpenFlow controller application that provides layer-based quality optimization. 
In section IV we present the general architecture, and in section V we show the algorithms used. 
Section VI shows experimental results.
In Section VII we come to a conclusion.

\section{Related work}
In our research we identified a need for separate forwarding actions per flow.
Wang, R. et al. \cite{wang2011openflow} follow the same principal.
The aim was to apply load-balancing during service discovery of web services 
when all requests had to be processed by a central component.
A problem occurs when each request is forwarded to a centralized component responsible 
for service discovery, which leads to network congestion. 
By using OpenFlow they managed to set separate forwarding actions for client requests which match 
certain criteria. These actions allow forwarding entities to redirect client requests to an 
appropriate service instance without interference from the centralized component. Even though 
the use case may be different, the principal is the same as the one applied in our research. 
When all traffic follows the same path, chances of congestion and packet loss 
occurring increase significantly. OpenFlow allows us to set up different paths for different flows, which 
improves the throughput of our QoS flows and also decrease the network load on links used by best 
effort traffic.

The idea to provide a minimal quality for SVC encoded videos originates from 
research described in \cite{egilmez2011scalable} and \cite{civanlar2010qos}. 
By routing the base layer as a lossless QoS flow, which means there is no packet loss, they can 
guarantee a minimal quality to the user. Ideally, the enhancement layers will also be delivered 
without packet loss or delay but when congestion occurs they demote the enhancement layers to best 
effort traffic and prioritize the base layer.

\begin{figure}[htb]
\centering
\includegraphics[width=\linewidth]{fig/SVC_OpenFlow_flow}
\caption{General problem}
\label{fig_sim}
\end{figure}

In our own research we define the base layer as lossless QoS traffic, which means no packet loss is allowed.
The enhancement layers are lossy QoS traffic, which means they may still follow a different 
route than best-effort traffic, but some packet loss is allowed.
These principals are shown in Figure \ref{fig_sim}.
We also enable fast failure recovery when a link fails, as described in \cite{sharma2011enabling}.
Also, we present experimental results, whereas other research only presents simulations.

\section{Design considerations}

In designing the OpenFlow application, we considered the following.
\begin{itemize}
	\item When high priority QoS flows are forwarded on the same link as best effort traffic, there will be packet loss in the best effort flow to prioritize the QoS flow.
	\item Metrics for QoS flows must be different than those used for best effort traffic. The shortest path and QoS path must be different to obtain good results. 
	If both follow the same path, the best effort traffic will experience more packet loss.
	\item Calculating a path for a new QoS flow does not change the best effort routing tables. The packet loss of best effort traffic caused by QoS flows must be considered during path selection.
	\item The length of alternative paths for QoS flows should still be kept as short as possible to optimize transfer time.
	\item Best effort traffic can be used as a reference to retrieve metrics from a switch or link, such as packet loss or to estimate the amount of traffic on that route.
\end{itemize}

These observations show that packet loss and path length are important in defining algorithms: 
the algorithm must find a balance between maintaining low packet loss of the best effort 
traffic and the delay of QoS flows, caused by using longer paths.

\section{Architecture}

To obtain the different SVC video streams, we use HTTP adaptive streaming.
This means that the video is split into different segments, and when a new segment starts playing, 
the following segment is requested using HTTP. 
If the next segment doesn't arrive in time, it is skipped.
In the case of SVC streaming the video stream is first split into different layers (base and enhancement layers).
These layers are then further split into segments of different size, but equal playtime.
If a segment of a layer does not arrive in time, it is skipped.

Each OpenFlow network contains at least one controller which is responsible for managing all the switches. 
An OpenFlow switch forwards packets 
to the controller if the packet does not match any forwarding entry in the forwarding table.
Matching is done on the longest prefix. Wildcards are allowed in the forwarding table.
A filter in the controller separates QoS flows from 
other traffic, based on the TCP port, after which the controller can choose a path for the different layers. 
When the controller calculates a path for a layer, the new forwarding rules are sent to the switches.

The different paths are calculated using Dijkstra. 
Depending on the current load of the network, a cost is given to each link.
The controller must be able to obtain the load any time from any switch in the network before calculating 
new flow tables. We implemented a monitor module on the controller to gather all network 
statistics. Besides polling for statistics, a switch also forwards events to the controller when 
they occur, such as a link failure. These events are also act upon by the controller, 
by recalculating a different path for every flow.

\section{Algorithms}

Because the base layer has to be delivered lossless, the weights given to each link depends 
on the load of the switches connected to these switches.
The minimum value a link can get is 1, so the hop count is used when the network experiences no load.
The pseudocode can be seen in Algorithm \ref{alg1}. This algorithm is executed on receiving a packet-in event 
from a switch with TCP port set to 8080.

\begin{algorithm}
\caption{Calculate route for base layer}
\label{alg1}
\begin{algorithmic}
\FORALL{$link$ in $linksInNetwork$}
\STATE $leftSwitch \leftarrow link.getSrc()$
\STATE $rightSwitch \leftarrow link.getDst()$
\STATE $loadLeft \leftarrow monitorModule.getLoad(LeftSwitch)$
\STATE $loadRight \leftarrow monitorModule.getLoad(rightSwitch)$
\STATE $totalLoad \leftarrow loadLeft + loadRight + 1$
\STATE $linkMap.put(link,weight)$
\ENDFOR
$Dijkstra(linkMap)$
\end{algorithmic}
\end{algorithm}

For the enhancement layers we followed the same tactic, but we also hold the route that the 
base layer is following into account. This means we gave a higher cost to the links that the base layer was using, 
thus ensuring the base layer is delivered lossless. The pseudocode for this algorithm can be seen in Algorithm \ref{alg2}.
It is executed on packet-in events with the TCP port of the packet set to 8081,8082 or 8083.
The constant extraLoad can be tweaked to render optimal performance. It must be set high enough to ensure 
the route the base layer is using is not used, unless the traffic on the other links is much higher.

\begin{algorithm}
\caption{Calculate route for base layer}
\label{alg2}
\begin{algorithmic}
\FORALL{$link$ in $linksInNetwork$}
\STATE $leftSwitch \leftarrow link.getSrc()$
\STATE $rightSwitch \leftarrow link.getDst()$
\STATE $loadLeft \leftarrow monitorModule.getLoad(LeftSwitch)$
\STATE $loadRight \leftarrow monitorModule.getLoad(rightSwitch)$
\STATE $totalLoad \leftarrow loadLeft + loadRight + 1$
\IF{$linksRouteBaseLayer.contains(link)$}
\STATE $totalLoad \leftarrow totalLoad + Constants.extraLoad()$
\ENDIF
\STATE $linkMap.put(link,weight)$
\ENDFOR
$Dijkstra(linkMap)$
\end{algorithmic}
\end{algorithm}

\section{Experimental results}

All of the experiments are performed on the iLab.t Virtual Wall at iMinds.
The Virtual Wall is derived from Emulab, developed at Utah university.
The Virtual Wall facility is a generic test environment for advanced network, 
distributed software and service evaluation, and supports scalability research.

The Virtual Wall facilities consist of 100 nodes 
(dual processor, dual core servers, 4x1 or 6x1 Gb/s interfaces per node)
interconnected via a non-blocking 1.5 Tb/s VLAN Ethernet switch enabling flexible creation of experiment topologies, 
and a display wall (20 monitors) for experiment visualization. 
Each server is connected with 4 or 6 gigabit Ethernet links to the switch. 
The Virtual Wall nodes can be assigned different functionalities ranging 
from terminal, server, network node, and impairment node. 
Each node is fully dedicated and under full control of one experiment, 
so the users can test all kind of operating system images and configurations \cite{virtual}.

We created the topology shown in Figure \ref{fig:topology}, on which we performed the experiments.
Each link has a 1 Gb connection. The switches are running OpenVSwitch instead of the reference user-space switch, 
because Floodlight won't work with the reference user-space switch.

\begin{figure}[htb]
\centering
\includegraphics[scale=0.4]{fig/topology}
\caption{Topology used}
\label{fig:topology}
\end{figure}

To measure the quality of the video we chose the PSNR value of the received video compared to the raw footage. 
The higher the PSNR value, the better the frames were preserved during transfer. 
A base layer has a low PSNR value, and every extra enhancement layer that is also received, 
increases this value. %In Figure \ref{fig:psnr} we show the PSNR value as a function of time.
Unfortunately, we did not have enough time to obtain results. 

\section{Conclusion}
In this paper we described a way to perform quality optimization of video streaming.
We do this by using HTTP adaptive streaming of an SVC encoded video stream, which is composed of a base layer and three enhancement layers.
Each layer is routed differently, depending on the required Quality of Service. 
The base layer is delivered as a lossless video stream. This means it is always delivered. It has the highest QoS priority.
The enhancement layers are considered lossy video streams, which means they may not always arrive, 
depending on the network load. We also implemented fast failure recovery. This means if a link fails, the routes get recalculated very fast, 
rendering an optimal performance.

In future work we will try to optimize the weights given to each link in the algorithm. 
This will give an even higher quality optimization.


%\subsection{Subsection Heading Here}
%Subsection text here\cite{IEEEexample:confwithpaper}.


%\subsubsection{Subsubsection Heading Here}
%Subsubsection text here.


% An example of a floating figure using the graphicx package.
% Note that \label must occur AFTER (or within) \caption.
% For figures, \caption should occur after the \includegraphics.
% Note that IEEEtran v1.7 and later has special internal code that
% is designed to preserve the operation of \label within \caption
% even when the captionsoff option is in effect. However, because
% of issues like this, it may be the safest practice to put all your
% \label just after \caption rather than within \caption{}.
%
% Reminder: the "draftcls" or "draftclsnofoot", not "draft", class
% option should be used if it is desired that the figures are to be
% displayed while in draft mode.
%
%\begin{figure}[!t]
%\centering
%\includegraphics[width=2.5in]{myfigure}
% where an .eps filename suffix will be assumed under latex, 
% and a .pdf suffix will be assumed for pdflatex; or what has been declared
% via \DeclareGraphicsExtensions.
%\caption{Simulation Results}
%\label{fig_sim}
%\end{figure}

% Note that IEEE typically puts floats only at the top, even when this
% results in a large percentage of a column being occupied by floats.


% An example of a double column floating figure using two subfigures.
% (The subfig.sty package must be loaded for this to work.)
% The subfigure \label commands are set within each subfloat command, the
% \label for the overall figure must come after \caption.
% \hfil must be used as a separator to get equal spacing.
% The subfigure.sty package works much the same way, except \subfigure is
% used instead of \subfloat.
%
%\begin{figure*}[!t]
%\centerline{\subfloat[Case I]\includegraphics[width=2.5in]{subfigcase1}%
%\label{fig_first_case}}
%\hfil
%\subfloat[Case II]{\includegraphics[width=2.5in]{subfigcase2}%
%\label{fig_second_case}}}
%\caption{Simulation results}
%\label{fig_sim}
%\end{figure*}
%
% Note that often IEEE papers with subfigures do not employ subfigure
% captions (using the optional argument to \subfloat), but instead will
% reference/describe all of them (a), (b), etc., within the main caption.


% An example of a floating table. Note that, for IEEE style tables, the 
% \caption command should come BEFORE the table. Table text will default to
% \footnotesize as IEEE normally uses this smaller font for tables.
% The \label must come after \caption as always.
%
%\begin{table}[!t]
%% increase table row spacing, adjust to taste
%\renewcommand{\arraystretch}{1.3}
% if using array.sty, it might be a good idea to tweak the value of
% \extrarowheight as needed to properly center the text within the cells
%\caption{An Example of a Table}
%\label{table_example}
%\centering
%% Some packages, such as MDW tools, offer better commands for making tables
%% than the plain LaTeX2e tabular which is used here.
%\begin{tabular}{|c||c|}
%\hline
%One & Two\\
%\hline
%Three & Four\\
%\hline
%\end{tabular}
%\end{table}


% Note that IEEE does not put floats in the very first column - or typically
% anywhere on the first page for that matter. Also, in-text middle ("here")
% positioning is not used. Most IEEE journals/conferences use top floats
% exclusively. Note that, LaTeX2e, unlike IEEE journals/conferences, places
% footnotes above bottom floats. This can be corrected via the \fnbelowfloat
% command of the stfloats package.






% conference papers do not normally have an appendix


% use section* for acknowledgement
%\section*{Acknowledgment}
%The authors would like to thank...





% trigger a \newpage just before the given reference
% number - used to balance the columns on the last page
% adjust value as needed - may need to be readjusted if
% the document is modified later
%\IEEEtriggeratref{8}
% The "triggered" command can be changed if desired:
%\IEEEtriggercmd{\enlargethispage{-5in}}

% references section

% can use a bibliography generated by BibTeX as a .bbl file
% BibTeX documentation can be easily obtained at:
% http://www.ctan.org/tex-archive/biblio/bibtex/contrib/doc/
% The IEEEtran BibTeX style support page is at:
% http://www.michaelshell.org/tex/ieeetran/bibtex/
%\bibliographystyle{IEEEtran}
% argument is your BibTeX string definitions and bibliography database(s)
%\bibliography{IEEEabrv,../bib/paper}
%
% <OR> manually copy in the resultant .bbl file
% set second argument of \begin to the number of references
% (used to reserve space for the reference number labels box)
\bibliographystyle{IEEEtran}
\bibliography{paper}



% that's all folks
\end{document}


